\documentclass[preprint,12pt, a4paper]{elsarticle}

\usepackage{geometry}
\geometry{
    a4paper,
    left=2cm,
    right=2cm,
    top=1.5cm,
    bottom=3cm
}
\usepackage{listings}
\usepackage{xcolor}

% Configure code highlighting
\lstset{
    language=Java,
    basicstyle=\ttfamily\footnotesize,
    keywordstyle=\color{blue}\bfseries,
    commentstyle=\color{green!60!black},
    stringstyle=\color{red},
    numbers=left,
    numberstyle=\tiny\color{gray},
    stepnumber=1,
    numbersep=5pt,
    backgroundcolor=\color{gray!10},
    showspaces=false,
    showstringspaces=false,
    showtabs=false,
    frame=single,
    rulecolor=\color{black},
    tabsize=2,
    captionpos=b,
    breaklines=true,
    breakatwhitespace=false,
    escapeinside={\%*}{*)},
    morekeywords={String, double, static, private, public, if, else, return, new}
}
\usepackage{graphicx}
\usepackage{setspace}
\usepackage{float}
\usepackage{hyperref}
\usepackage[utf8]{inputenc}
\usepackage[english]{babel}
\usepackage{framed}
\usepackage{enumitem}
\usepackage{pdfpages}
\usepackage{tablefootnote}
\usepackage{amssymb}
\usepackage{subfig}
\usepackage{multicol}
\usepackage{booktabs}
\setlength{\parindent}{0pt}

\journal{AgroTrace-MS}

\begin{document}
\renewcommand{\labelenumii}{\arabic{enumi}.\arabic{enumii}}

\begin{frontmatter} 
\title{AgroTrace-MS: Event-Driven Microservices Architecture for Crop Monitoring and Smart Irrigation}

\author[label1]{El Badouri Abderrahmane}
\author[label1]{Ait Abderrahamne Hind}
\author[label1]{Adnane Mohamed Amine}
\author[label1]{El Mourabit Asmaa}
\author[label1]{Ouqelli Yassmine}


\address[label1]{Moroccan School of Engineering Sciences (EMSI), Computer and Network Engineering Department (IIR), 5IIR, Marrakech, Morocco}

\begin{abstract}
Modern agriculture faces a dual constraint: optimizing yields to feed a growing population while preserving increasingly scarce water resources. AgroTrace-MS is a modular open-source platform designed to address this challenge. It deploys a distributed microservices architecture to orchestrate massive IoT data ingestion, drone imagery analysis for early disease detection, and climate-based irrigation planning. The system uses an Apache Kafka event bus to ensure decoupling and resilience, and relies on Docker containerization to ensure deployment reproducibility. By integrating real-time sensor data, predictive analytics, and crop-specific thresholds, the system delivers localized guidance tailored to each parcel's geolocation and crop type. Preliminary evaluations suggest that AgroTrace-MS may reduce water waste by up to 25\% and enable early disease detection, depending on crop sensitivity and regional conditions. This initiative contributes to improved resource management and supports global efforts in climate-resilient, sustainable agriculture.
\end{abstract}

\begin{keyword}
Precision Agriculture \sep Microservices \sep Apache Kafka \sep IoT \sep Deep Learning \sep Docker \sep Event-Driven Architecture
\end{keyword}

\end{frontmatter}

%% ==============================================================================
%% METADATA TABLE
%% ==============================================================================
\section*{Metadata}
\begin{table}[!ht]
\centering
\begin{tabular}{|l|p{7.5cm}|p{7.5cm}|}
\hline
\textbf{Nr.} & \textbf{Code metadata description} & \textbf{Metadata} \\
\hline
C1 & Current code version & v1.0.3 \\
\hline
C2 & Permanent link to code/repository used for this code version & \url{https://github.com/Abderrahmane-124/AgroTrace-MS} \\
\hline
C3 & Permanent link to Reproducible Capsule & N/A \\
\hline
C4 & Legal Code License & MIT License \\
\hline
C5 & Code versioning system used & Git \\
\hline
C6 & Software code languages, tools, and services used & Python, FastAPI, Java, Spring Boot, React.js, Apache Kafka, Docker \\
\hline
C7 & Compilation requirements, operating environments \& dependencies & Docker Engine v24+, Docker Compose v2+, Python 3.10+, Node.js 18+ \\
\hline
C8 & Link to developer documentation/manual & \url{https://github.com/Abderrahmane-124/AgroTrace-MS/blob/main/MICROSERVICES.md} \\
\hline
C9 & Support email for questions & abderrahmaneelbadouri124@gmail.com \\
\hline
\end{tabular}
\label{codeMetadata} 
\end{table}

%% ==============================================================================
%% SECTION 1: MOTIVATION AND SIGNIFICANCE
%% ==============================================================================
\section{Motivation and significance}

Agriculture faces growing vulnerability due to climate variability and extreme events—droughts, floods, heatwaves, and pest outbreaks—that cause substantial economic losses estimated at over \$12 billion annually~\cite{fao2022}. Smallholder farmers, producing about one-third of the world's food, are disproportionately impacted by these disruptions because they have limited access to reliable weather forecasts and adaptive technologies~\cite{ricciardi2018}. For example, in arid and semi-arid regions, prolonged droughts have led to 20–40\% declines in cereal yields over the past decade~\cite{worldbank2021}, exacerbating food insecurity.

AgroTrace-MS responds to these challenges by delivering real-time monitoring and actionable, crop-specific recommendations based on localized sensor data and weather forecasts. Unlike conventional tools that focus on single variables such as soil moisture or irrigation, AgroTrace-MS integrates real-time IoT data with adaptive decision models tailored to crop type and location. This enables the platform to provide optimized irrigation advice and early disease detection through drone imagery analysis, enhancing both anticipatory capacity and adaptive response~\cite{iot_agri}.

Reactive strategies alone are often insufficient for preventing major yield losses. AgroTrace-MS shifts to a proactive approach, issuing alerts based on predictive indicators so that farmers can adjust practices before critical thresholds are breached. Advance warnings, for example, enable timely adoption of water-saving or protective measures, reducing potential losses by up to 25\%~\cite{worldbank2021}.

As global population nears 9.7 billion by 2050~\cite{ipcc2022}, protecting agricultural productivity under climate stress becomes imperative. AgroTrace-MS supports this through precision monitoring and localized interventions that help farmers maintain yields and use inputs efficiently, aligning with Sustainable Development Goal 2 (Zero Hunger).

To address the limitations of existing platforms, AgroTrace-MS combines event-driven architecture, computer vision diagnostics, and broad accessibility. Its core features include real-time sensor data ingestion through Apache Kafka, deep learning-based plant disease detection using the PlantVillage dataset~\cite{plant_disease}, and a rule-based decision engine for irrigation optimization. The platform incorporates authoritative crop-climate thresholds and maintains robustness through containerized deployment. Its holistic feature set—including weather integration, disease alerts, predictive analytics, and GIS visualization—positions AgroTrace-MS as a comprehensive solution for farmers in climate-vulnerable regions (Table~\ref{tab:tools_comparison}).

%% ==============================================================================
%% SECTION 2: SOFTWARE DESCRIPTION
%% ==============================================================================
\section{Software description}

This project delivers a user-friendly platform focused on real-time decision-making and proactive risk management in agriculture. By analyzing location-specific sensor data and integrating crop-specific recommendations, the system enables farmers to make informed decisions, protect yields, and minimize water consumption. Advanced analytics and predictive modeling provide real-time updates and actionable insights, helping users adapt to adverse conditions, optimize resource allocation, and build resilience against climate change. Ultimately, AgroTrace-MS supports sustainable farming and food security.

\subsection{Software Architecture}

The AgroTrace-MS platform is built with a modern, flexible, and robust software design that meets the needs of agricultural resilience and fast decision-making. This section delineates the architectural selections and their justifications, illustrating their contributions to system efficacy and user satisfaction (see Fig.~\ref{fig:architecture}).

\begin{enumerate}
    \item Overview of the Architecture
    
    AgroTrace-MS employs a microservices-based architecture to enhance scalability, modularity, and maintainability. Each component, from the web dashboard to the AI services, operates independently but communicates seamlessly through Apache Kafka message queues. The architecture is containerized using Docker~\cite{docker}, ensuring consistent development, testing, and production environments. This modular design allows for efficient updates and minimal downtime, making the platform flexible and reliable for diverse user needs.
    
    \item Backend Infrastructure
    
    The backend infrastructure is designed to be robust, secure, and scalable to handle real-time agricultural data processing. Several key technologies and frameworks are utilized to achieve this goal:
    
    \begin{itemize}
        \item FastAPI Framework: The majority of microservices (MS0, MS2-MS6) are built with FastAPI~\cite{fastapi} due to its excellent async capabilities and performance for data science workloads. Its automatic API documentation and type validation make it ideal for rapid development.
        \item Spring Boot Framework: The ingestion service (MS1) is built with Spring Boot~\cite{springboot} for its robustness in handling high-throughput data streams and enterprise-grade reliability.
        \item Apache Kafka: Deployed in KRaft mode, Kafka serves as the central nervous system for event-driven communication between all microservices, enabling temporal decoupling and backpressure management.
    \end{itemize}
    
    \item Frontend Composition
    
    The frontend is designed to ensure a seamless user experience with modern frameworks and best practices:
    
    \begin{itemize}
        \item React.js for Web Client: The web interface uses React.js for component-based architecture, ensuring a responsive and maintainable user experience. It integrates Leaflet.js for interactive GIS visualization.
        \item Real-time Updates: WebSocket connections enable live sensor data updates without page refresh.
        \item Responsive Design: The interface is designed to be accessible across various devices, including desktops and tablets.
    \end{itemize}
    
    \item Database Management
    
    Effective database management is crucial for ensuring data integrity, security, and performance. The AgroTrace-MS platform employs a polyglot persistence strategy:
    
    \begin{itemize}
        \item TimescaleDB: Optimized for high-frequency time-series IoT sensor data with automatic data retention policies.
        \item MinIO: S3-compatible object storage for drone imagery and large binary files.
        \item PostgreSQL/PostGIS: Relational database with geospatial extensions for parcel boundaries and zone management.
    \end{itemize}
    
    \item Communication and Data Flow
    
    The platform employs Apache Kafka topics to facilitate seamless asynchronous communication between microservices. Data flows through well-defined topics:
    \begin{itemize}
        \item \texttt{sensor-raw}: Raw sensor data from simulator to ingestion
        \item \texttt{sensor-clean}: Validated data from ingestion to preprocessing
        \item \texttt{sensor-processed}: Aggregated data for downstream analysis
        \item \texttt{vision-results}: Disease detection outputs from the vision service
        \item \texttt{water-forecast}: Predictions for the rules engine
        \item \texttt{recommendations}: Final irrigation recommendations
    \end{itemize}
    
    \item Deployment and Quality Assurance
    
    AgroTrace-MS is deployed as a containerized stack: all seven microservices communicate through Kafka, with each service running in its own Docker container. The stack includes TimescaleDB for sensor data, MinIO for images, and PostGIS for geospatial data.
    
    Docker~\cite{docker} images are used consistently across development, staging, and production, with Docker Compose for orchestration. System health is monitored via dedicated endpoints, and each service includes comprehensive logging for debugging and auditing.
    
    \item Innovative Components
    
    To enhance user experience and decision-making capabilities, the AgroTrace-MS platform integrates several innovative features:
    
    \begin{itemize}
        \item Deep Learning Disease Detection: Implemented using a CNN model trained on the PlantVillage dataset~\cite{plant_disease}, the vision service analyzes drone imagery to detect plant diseases before visible symptoms appear, enabling early intervention.
        \item Rules Engine: A configurable decision engine evaluates sensor data against crop-specific thresholds to generate actionable recommendations for irrigation and alerts.
        \item Predictive Water Forecasting: Time-series models (Prophet/LSTM) predict future water requirements based on historical sensor data and weather patterns.
        \item GIS Dashboard: Interactive map visualization with Leaflet.js displays parcel health status, sensor readings, and recommended actions in real-time.
    \end{itemize}
\end{enumerate}

Figure~\ref{fig:schema_simplifie} presents a simplified overview of the AgroTrace-MS architecture, organized in four distinct layers. At the bottom, Apache Kafka (KRaft mode) serves as the central event bus, enabling asynchronous communication between all microservices. The middle layer contains the seven specialized microservices (MS0--MS6): the Sensor Simulator (MS0) generates IoT data, the Ingestion service (MS1) validates and persists raw readings, the Preprocessing service (MS2) handles data cleaning and anomaly detection, the VisionPlante service (MS3) performs AI-based disease detection, the Water Forecasting service (MS4) predicts irrigation needs, the AgriRules engine (MS5) applies crop-specific thresholds, and the Irrigation Recommendation service (MS6) generates actionable irrigation plans. The storage layer includes four specialized databases: TimescaleDB for time-series sensor data, MinIO for drone imagery, PostgreSQL for irrigation plans, and PostGIS for geospatial parcel data. Finally, the presentation layer exposes REST APIs consumed by the React.js web application (MS7), which provides the GIS dashboard interface.

\begin{figure}[H]
    \centering
    \includegraphics[width=0.95\textwidth]{Schema-global-simplifie.png}
    \caption{Schéma Global Simplifié de l'Architecture AgroTrace-MS}
    \label{fig:schema_simplifie}
\end{figure}

\begin{figure}[H]
    \centering
    \includegraphics[width=1\textwidth]{architecture.png}
    \caption{Schéma Global detaillée de l'Architecture AgroTrace-MS}
    \label{fig:architecture}
\end{figure}

\subsection{Software functionalities}

The AgroTrace-MS platform offers a comprehensive set of user-focused features to enhance agricultural resilience and manage resource optimization. Its functionalities are designed to optimize user experience, enable timely decisions, and support effective crop management:

\begin{enumerate}
    \item \textbf{MS0 - Sensor Data Simulation:}
    The simulator service generates realistic test data from historical sensor readings at configurable intervals. It reads CSV data files and injects JSON messages into Kafka topics, enabling development and load testing without physical sensors (see Fig.~\ref{fig:bpmn_ms0}).
    
    \begin{figure}[H]
        \centering
        \includegraphics[width=0.9\linewidth]{Microservice0.png}
        \caption{BPMN Workflow - MS0 Sensor Simulator}
        \label{fig:bpmn_ms0}
    \end{figure}
    
    \item \textbf{MS1 - Data Ingestion and Validation:}
    The ingestion service consumes raw sensor messages from Kafka, validates data format and ranges, normalizes values, and persists clean data to TimescaleDB. Invalid records are logged for review (see Fig.~\ref{fig:bpmn_ms1}).
    
    \begin{figure}[H]
        \centering
        \includegraphics[width=0.9\linewidth]{Microservice1.png}
        \caption{BPMN Workflow - MS1 Data Ingestion}
        \label{fig:bpmn_ms1}
    \end{figure}
    
    \item \textbf{MS2 - Data Preprocessing and Anomaly Detection:}
    Raw sensor data undergoes preprocessing including outlier detection using statistical methods (IQR, Z-score) and temporal aggregation. Anomalous readings are flagged for review, ensuring data quality for downstream analysis (see Fig.~\ref{fig:bpmn_ms2}).
    
    \begin{figure}[H]
        \centering
        \includegraphics[width=0.9\linewidth]{Microservice2.png}
        \caption{BPMN Workflow - MS2 Preprocessing}
        \label{fig:bpmn_ms2}
    \end{figure}
    
    \item \textbf{MS3 - Plant Disease Detection:}
    The vision service analyzes drone imagery using a CNN model trained on PlantVillage. Detection results include disease type, confidence score, and affected area location. Early detection enables farmers to take preventive measures before widespread crop damage occurs (see Fig.~\ref{fig:bpmn_ms3}).
    
    \begin{figure}[H]
        \centering
        \includegraphics[width=0.9\linewidth]{Microservice3.png}
        \caption{BPMN Workflow - MS3 Vision Service}
        \label{fig:bpmn_ms3}
    \end{figure}
    
    \item \textbf{MS4 - Water Demand Forecasting:}
    Predictive models analyze historical sensor data and weather patterns to forecast future water requirements. Forecasts are generated for configurable time horizons (24h, 48h, 7 days), enabling proactive irrigation planning (see Fig.~\ref{fig:bpmn_ms4}).
    
    \begin{figure}[H]
        \centering
        \includegraphics[width=0.9\linewidth]{Microservice4.png}
        \caption{BPMN Workflow - MS4 Water Forecasting}
        \label{fig:bpmn_ms4}
    \end{figure}
    
    \item \textbf{MS5 - Rules Engine and Alert System:}
    Automated alerts notify users about hazards such as water stress, disease detection, or sensor anomalies. The rules engine evaluates crop-specific thresholds and generates severity-classified recommendations (LOW, MEDIUM, HIGH). Users can access alert history for review and planning (see Fig.~\ref{fig:bpmn_ms5}).
    
    \begin{figure}[H]
        \centering
        \includegraphics[width=0.9\linewidth]{Microservice5.png}
        \caption{BPMN Workflow - MS5 Rules Engine}
        \label{fig:bpmn_ms5}
    \end{figure}
    
    \item \textbf{MS6 - Irrigation Recommendations:}
    The system provides tailored irrigation plans based on current soil conditions, weather forecasts, and crop-specific water requirements. Plans include recommended timing, duration, and volume. Farmers can validate or manually adjust recommendations before execution (see Fig.~\ref{fig:bpmn_ms6}).
    
    \begin{figure}[H]
        \centering
        \includegraphics[width=0.9\linewidth]{Microservice6.png}
        \caption{BPMN Workflow - MS6 Irrigation Recommendation}
        \label{fig:bpmn_ms6}
    \end{figure}
    
    \item \textbf{MS7 - GIS Dashboard:}
    An interactive map interface displays all monitored parcels with color-coded health indicators. Users can click on parcels to view detailed sensor data, active alerts, and irrigation schedules. The dashboard supports filtering by crop type, alert status, and date range.
\end{enumerate}

\subsection{Decision Engine Architecture}

The AgroTrace-MS decision engine combines rule-based logic with predictive analytics to provide both transparent explanations and actionable insights. At runtime, sensor data and forecasts are aggregated by the rules engine (MS5), which evaluates crop-specific thresholds and generates recommendations.

The engine consists of two processing layers:

\begin{enumerate}
    \item \textbf{Rule-based layer}: Using configurable crop tolerance profiles (min/max temperature, soil moisture, humidity), it calculates deviations from optimal ranges and generates severity scores (LOW/MEDIUM/HIGH). Targeted recommendations are produced from an auditable knowledge base, ensuring explainability for each suggested action.
    
    \item \textbf{Predictive layer}: The forecasting service (MS4) runs time-series models to predict future water requirements. When predictions indicate upcoming stress conditions, proactive recommendations are generated before thresholds are actually breached.
\end{enumerate}

The orchestration routes sensor data through validation, preprocessing, and analysis stages, merging outputs into actionable irrigation plans for the dashboard. This modular design allows updates to threshold configurations, forecasting models, or recommendation templates independently.

%% ==============================================================================
%% SECTION 3: ILLUSTRATIVE EXAMPLES
%% ==============================================================================
\section{Illustrative Examples}

To better understand how AgroTrace-MS assists farmers in optimizing irrigation and detecting crop issues, we present several illustrative scenarios that showcase its practical applicability. These instances demonstrate how the platform delivers real-time monitoring, tailored suggestions, and actionable insights to facilitate agricultural decision-making.

\subsection{Real-Time Parcel Monitoring}

Agricultural productivity requires continuous monitoring of soil and environmental conditions. The GIS dashboard provides an overview of all monitored parcels, with color-coded indicators showing current health status (see Fig.~\ref{fig:dashboard}).

When a farmer accesses the dashboard, they can immediately identify parcels requiring attention:
\begin{itemize}
    \item Green parcels indicate optimal conditions
    \item Yellow parcels show moderate stress requiring monitoring
    \item Red parcels indicate critical conditions requiring immediate action
\end{itemize}

The dashboard also displays key performance indicators (KPIs) including average soil moisture, temperature trends, and active alert counts across all parcels.

\begin{figure}[H]
    \centering
    \includegraphics[width=0.95\textwidth]{WhatsApp Image 2025-12-24 at 23.35.38.jpeg}
    \caption{Global Dashboard View with Key Performance Indicators (KPIs)}
    \label{fig:dashboard}
\end{figure}

\subsection{Smart Irrigation Planning}

When a user clicks on a specific parcel, a detailed view displays current sensor readings and enables irrigation planning (see Fig.~\ref{fig:irrigation}).

The irrigation interface provides:
\begin{itemize}
    \item Current soil moisture and temperature readings
    \item 7-day weather forecast integration
    \item AI-recommended irrigation schedule with timing and volume
    \item Historical irrigation records
    \item Manual override options for farmer control
\end{itemize}

For example, when soil moisture drops below the crop-specific threshold (e.g., 30\% for tomatoes), the system automatically generates a recommended irrigation plan. The farmer can review the recommendation, adjust parameters if needed, and approve the plan for execution.

\begin{figure}[H]
    \centering
    \includegraphics[width=0.9\linewidth]{WhatsApp Image 2025-12-24 at 23.35.38 (2).jpeg}
    \caption{Irrigation Control and Planning Interface}
\end{figure}

\subsection{Alert Management and Actions}

The lateral panel displays all active alerts and recommended actions generated by the rules engine (see Fig.~\ref{fig:alerts}). Alerts are categorized by type and severity:

\begin{itemize}
    \item \textbf{Disease Alerts}: Generated when the vision service detects plant diseases in drone imagery. Includes disease type, confidence score, and recommended treatment.
    \item \textbf{Water Stress Alerts}: Triggered when soil moisture falls below critical thresholds. Includes recommended irrigation volume and timing.
    \item \textbf{Sensor Anomaly Alerts}: Flagged when preprocessing detects unusual sensor readings, prompting equipment inspection.
\end{itemize}

Each alert includes a detailed action recommendation that farmers can accept, modify, or dismiss based on their domain expertise.

\begin{figure}[H]
    \centering
    \includegraphics[width=0.9\linewidth]{WhatsApp Image 2025-12-24 at 23.35.38 (1).jpeg}
    \caption{Alert Details and Action Recommendations Panel}
    \label{fig:alerts}
\end{figure}

%% ==============================================================================
%% SECTION 4: IMPACT
%% ==============================================================================
\section{Impact}

The integration of real-time monitoring, AI-based analytics, and crop-specific recommendations in AgroTrace-MS strengthens both agricultural resilience and decision-making. By combining event-driven architecture, computer vision, and IoT sensors, the platform improves irrigation efficiency, disease detection, and resource optimization, supporting precision farming and sustainable agriculture~\cite{iot_agri}.

In practice, AgroTrace-MS empowers farmers to anticipate and respond to crop stress through real-time alerts and tailored advice, enabling efficient input use and reducing both costs and potential losses. The GIS dashboard provides intuitive visualization of parcel status, making scientific insights accessible even to farmers with limited technical background.

From a sustainability perspective, AgroTrace-MS supports water conservation and food security by promoting data-driven irrigation management. It offers actionable insights on soil conditions, water requirements, and disease prevention, helping farmers reduce water waste and adopt conservation practices. The platform's open-source nature and containerized deployment also facilitate adoption by agricultural cooperatives and research institutions.

\subsection{Comparison with Existing Tools}

Several existing tools support precision agriculture by offering sensor integration and agronomic recommendations. However, AgroTrace-MS differentiates itself through its unique combination of event-driven architecture, AI-powered diagnostics, and open-source accessibility.

\begin{table}[!h]
\centering
\caption{Comparative analysis of agricultural monitoring tools.}
\label{tab:tools_comparison}
\scriptsize
\begin{tabular}{|@{}p{3.5cm}|c|c|c|c|c|}
\hline
\textbf{Feature} & \textbf{AgroTrace-MS} & \textbf{CropX} & \textbf{Climate FieldView} & \textbf{FarmLogs} \\ \hline
Real-Time Sensor Integration & Yes & Yes & Limited & Limited \\ \hline
Event-Driven Architecture & Yes & No & No & No \\ \hline
AI Disease Detection & Yes & No & No & No \\ \hline
Open Source & Yes & No & No & No \\ \hline
Irrigation Optimization & Yes & Yes & Limited & Limited \\ \hline
GIS Dashboard & Yes & Limited & Yes & No \\ \hline
Docker Containerization & Yes & Unknown & Unknown & Unknown \\ \hline
Kafka Integration & Yes & No & No & No \\ \hline
Focus on Smallholders & High & Low & Moderate & High \\ \hline
\end{tabular}
\end{table}

As detailed in Table~\ref{tab:tools_comparison}, tools like CropX and Climate FieldView focus on irrigation optimization and yield forecasting. While CropX integrates soil monitoring and recommendations, Climate FieldView emphasizes data-driven insights. Yet both systems primarily use proprietary architectures and lack open-source accessibility.

AgroTrace-MS unifies real-time sensor integration, event-driven processing, and AI diagnostics in an open-source package. Its containerized deployment ensures reproducibility across different environments, making it suitable for both research and production use cases.

%% ==============================================================================
%% SECTION 5: FUTURE DIRECTIONS
%% ==============================================================================
\section{Future Directions}

To ensure the platform's robustness, scalability, and applicability, AgroTrace-MS will undergo collaborative validation. While the current implementation demonstrates strong potential for agricultural optimization, extended testing across diverse agro-ecological zones and crop types remains essential to refine predictive models and thresholds.

A key direction is field validation with agricultural cooperatives, where farmers can provide feedback on recommendation accuracy and usability. This participatory approach enables real-time calibration of crop-specific thresholds based on local conditions and expert knowledge.

Future developments will include:
\begin{itemize}
    \item \textbf{Mobile Application}: Native Android/iOS apps for in-field access to alerts and irrigation controls
    \item \textbf{Extended Crop Models}: Training disease detection models for additional crops beyond the current PlantVillage dataset
    \item \textbf{Weather API Integration}: Direct integration with meteorological services for enhanced forecast accuracy
    \item \textbf{Actuator Control}: Direct integration with irrigation hardware for automated water delivery
    \item \textbf{Multi-tenant Support}: Enabling multiple farms/cooperatives on a single platform instance
\end{itemize}

%% ==============================================================================
%% SECTION 6: QUALITY ASSURANCE
%% ==============================================================================
\section{Quality Assurance}

Quality assurance was conducted across all microservices to ensure reliability, maintainability, and correct functionality. The evaluation focused on architecture validation, code quality, and deployment consistency.

\begin{table}[h!]
\centering
\scriptsize
\caption{Quality metrics for AgroTrace-MS components}
\label{tab:qa_metrics}
\begin{tabular}{|l|c|c|c|c|}
\hline
\textbf{Metric} & \textbf{Backend (Java)} & \textbf{Backend (Python)} & \textbf{Frontend} \\ \hline
Architecture Validation & UML + BPMN & UML + BPMN & Component Diagram \\ \hline
Containerization & Docker & Docker & Docker \\ \hline
API Documentation & Swagger & FastAPI Docs & N/A \\ \hline
Code Quality & SonarQube Ready & Pylint Ready & ESLint Ready \\ \hline
\end{tabular}
\end{table}

\subsection{Architecture Validation}

The system architecture was validated through:
\begin{itemize}
    \item \textbf{UML Diagrams}: Class diagrams define the static structure of each microservice
    \item \textbf{BPMN Process Diagrams}: Document the dynamic behavior and data flow across services
    \item \textbf{Integration Testing}: End-to-end tests validate the complete Kafka-based data pipeline
\end{itemize}

\subsection{Jenkins CI/CD Pipeline}

AgroTrace-MS implements a comprehensive CI/CD pipeline using Jenkins, automating the build and deployment workflow. The pipeline is defined in a declarative \texttt{Jenkinsfile} at the repository root and leverages Docker-in-Docker (DinD) for containerized builds.

\begin{enumerate}
    \item \textbf{Checkout Stage}: Retrieves the latest source code from the Git repository
    
    \item \textbf{Build Docker Images}: Parallel execution builds Docker images for all nine microservice components:
    \begin{itemize}
        \item MS0--MS6: Individual backend services
        \item MS7 Backend: Dashboard API service
        \item MS7 Frontend: React.js web application
    \end{itemize}
    
    \item \textbf{Deploy to Production}: For the \texttt{main} branch, executes production deployment using \texttt{docker compose up -d}
\end{enumerate}

The pipeline includes post-execution hooks for cleanup (\texttt{docker system prune}) and status notifications for success or failure scenarios. Figure~\ref{fig:jenkins} shows the successful execution of the CI/CD pipeline with all nine microservices built in parallel.

\begin{figure}[H]
    \centering
    \includegraphics[width=0.75\textwidth]{jenkins-pipeline.png}
    \caption{Jenkins CI/CD Pipeline - Parallel Docker Build of All Microservices}
    \label{fig:jenkins}
\end{figure}

\subsection{Containerization and Deployment}

All services are containerized with Docker and orchestrated via Docker Compose:
\begin{itemize}
    \item Consistent environments across development, staging, and production
    \item Health check endpoints for each service
    \item Centralized logging for debugging and monitoring
    \item Single-command deployment: \texttt{docker-compose up -d}
\end{itemize}

%% ==============================================================================
%% SECTION 7: CONCLUSIONS
%% ==============================================================================
\section{Conclusions}

AgroTrace-MS represents a promising advancement in precision agriculture by combining real-time IoT monitoring with AI-powered disease detection and intelligent irrigation recommendations. The current implementation leverages event-driven architecture through Apache Kafka to deliver scalable, resilient data processing across seven specialized microservices.

Key contributions of this work include:
\begin{itemize}
    \item \textbf{Scalability}: The microservices architecture with Kafka enables handling increasing sensor loads without system redesign
    \item \textbf{Maintainability}: Strict separation of responsibilities (MS0-MS6) facilitates independent updates and debugging
    \item \textbf{Intelligence}: Joint integration of computer vision (MS3) and time-series forecasting (MS4) provides comprehensive decision support
    \item \textbf{Reproducibility}: Complete containerization ensures consistent deployments across environments
    \item \textbf{Accessibility}: Open-source licensing enables adoption by research institutions and agricultural cooperatives
\end{itemize}

In the face of escalating climate uncertainty and water scarcity, tools like AgroTrace-MS can play a crucial role within the broader ecosystem of agricultural decision-support systems. Its open-source architecture and containerized deployment foster inclusivity and reproducibility.

Future iterations of the platform will benefit from field validation with real agricultural operations, enabling refinement of thresholds and recommendations based on practical feedback. This community-driven approach will support the evolution of AgroTrace-MS as a globally adaptive, data-informed agricultural resilience tool.

AgroTrace-MS exemplifies how integrated digital tools can operationalize precision agriculture strategies, supporting sustainable farming and food security for a growing global population.

%% ==============================================================================
%% BIBLIOGRAPHY
%% ==============================================================================
\bibliographystyle{unsrt}
\begin{thebibliography}{10}

\bibitem{fao2022}
FAO. (2022).
\textit{The State of Food and Agriculture 2022}.
Food and Agriculture Organization of the United Nations.

\bibitem{ricciardi2018}
Ricciardi, V., et al. (2018).
How much of the world's food do smallholders produce?
\textit{Global Food Security}, 17, 64-72.

\bibitem{worldbank2021}
World Bank. (2021).
\textit{Climate-Smart Agriculture Investment Plans}.
World Bank Publications.

\bibitem{iot_agri}
Ray, P. P. (2017).
Internet of things for smart agriculture: Technologies, practices and future direction.
\textit{Journal of Ambient Intelligence and Humanized Computing}, 8(4), 395-420.

\bibitem{ipcc2022}
IPCC. (2022).
\textit{Climate Change 2022: Impacts, Adaptation and Vulnerability}.
Cambridge University Press.

\bibitem{plant_disease}
Mohanty, S. P., Hughes, D. P., \& Salathé, M. (2016).
Using deep learning for image-based plant disease detection.
\textit{Frontiers in Plant Science}, 7, 1419.

\bibitem{docker}
Merkel, D. (2014).
Docker: lightweight linux containers for consistent development and deployment.
\textit{Linux Journal}, 2014(239), 2.

\bibitem{fastapi}
Ramírez, S. (2018).
FastAPI: Modern, fast web framework for building APIs with Python.
\url{https://fastapi.tiangolo.com/}

\bibitem{springboot}
Pivotal. (2014).
Spring Boot Reference Documentation.
\url{https://spring.io/projects/spring-boot}

\bibitem{kafka}
Kreps, J., Narkhede, N., \& Rao, J. (2011).
Kafka: A Distributed Messaging System for Log Processing.
\textit{Proceedings of the NetDB}.

\end{thebibliography}

\end{document}
